\documentclass{article}
\usepackage[utf8]{inputenc}
\usepackage{cite}
\usepackage{tikz}
\usepackage{amsfonts}
\usepackage{amsmath}
\usepackage{bbm}
\usepackage{dsfont}
\usepackage{amsthm}
\usepackage{amssymb}
\usepackage{relsize}
\usepackage{csquotes}
\usepackage{graphicx}
\usepackage{epstopdf}
\usepackage{xcolor}
%\usepackage{fullpage}
\usepackage[utf8]{inputenc}
\usepackage[hidelinks]{hyperref}
\hypersetup{colorlinks=true,linkcolor=blue,citecolor=blue}
\usepackage{verbatim}
\setlength{\parindent}{0cm}
\newcommand{\dd}{\hspace{0.5ex}\text{d}}
\newcommand{\R}{\mathbb{R}}
\theoremstyle{plain}
\newtheorem{theorem}{Theorem}
\newtheorem{lemma}[theorem]{Lemma}
\newtheorem{cor}[theorem]{Corollary}
\newtheorem{prop}[theorem]{Proposition}
\theoremstyle{definition}
\newtheorem{conj}{Conjecture}
\newtheorem*{notation}{Notation}
\newtheorem{assume}{Assumption}
\newtheorem{exercise}[theorem]{Exercise}
\newtheorem{ex}[theorem]{Example}
\newtheorem{defn}[theorem]{Definition}
\newtheorem{rk}[theorem]{Remark}

%%%% PREAMBLE FROM ROBIN EVANS %%%%%
%\usepackage{amsthm}
%\usepackage{amsmath}
%\usepackage{amssymb}

\usepackage{algorithm2e}

\usepackage[normalem]{ulem}

%\usepackage[round]{natbib}

%\usepackage{a4wide}
\usepackage{bbm}
%\usepackage{graphicx}
%\usepackage{tikz}
\usetikzlibrary{positioning}

\usepackage{longtable}

\usepackage{caption}
\captionsetup[table]{
	skip=-5pt,
	format=plain,
	labelsep=newline,
	justification=centering,
	font=small,
	labelfont=sc,
	textfont=it
}
\captionsetup[longtable]{
	skip=5pt,
	format=plain,
	labelsep=newline,
	justification=centering,
	font=small,
	labelfont=sc,
	textfont=it
}
\captionsetup[figure]{
	format=plain,
	%labelsep=newline,
	justification=justified,
	font=small,
	labelfont=sc,
	textfont=it
}



%%%%%%%%%%%%%%%%%%%%%%%%%%%%%%%%%%%%%%%%%
%\newcommand{\RR}{\mathbbm{R}}
%\newcommand{\ZZ}{\mathbbm{Z}}
\DeclareMathOperator{\interior}{int}
\DeclareMathOperator{\conv}{conv}


%%%%%%%%%%%%%%%%%%%%%%%%%%%%%%%%%%%%%%%%%%

%%%%%%%%%%%%%%%%%%%%%%%%%%%%%%%%%%%%%
\theoremstyle{plain} % Heading is bold, text italic.
%\newtheorem{theorem}{Theorem}
%\newtheorem{lemma}[theorem]{Lemma}
%\newtheorem{proposition}[theorem]{Proposition}
%\newtheorem{corollary}[theorem]{Corollary}
%\newtheorem{conjecture}{Conjecture}
%\newtheorem*{theorem}{Theorem~1.1}
%\newtheorem*{cor2}{Corollary 1.2}
%\newtheorem*{them3}{Theorem~1.3}

%\theoremstyle{definition}  % Heading is bold, text is roman
%\newtheorem{definition}{Definition}
%\newtheorem{example}{Example}

%\theoremstyle{remark}  % Heading is italic, text is roman
%\newtheorem*{remark}{Remark}
%\newtheorem*{note}{Note}
%\newtheorem{claim}{Claim}
%%%%%%%%%%%%%%%%%%%%%%%%%%%%%%%%%%%%%%%%%%

%%%%%%%%%%%%%%%%%%%%%%%%%%%%%%%%%%%%%%%
% Shortcuts (add your own...):
%\renewcommand{\P}{{\mathcal P}}
%\newcommand{\F}{{\mathcal F}}
%\newcommand{\G}{{\mathcal G}}
%\newcommand{\e}{{\mathbf e}}
%\newcommand{\m}{{\mathbf m}}
%\renewcommand{\v}{{\mathbf v}}
%\renewcommand{\t}{{\mathbf t}}
%\newcommand{\x}{{\mathbf x}}
%\renewcommand{\u}{{\mathbf u}}
%\newcommand{\w}{{\mathbf w}}
%\newcommand{\y}{{\mathbf y}}
%\renewcommand{\a}{{\mathbf a}}
%\renewcommand{\b}{{\mathbf b}}
%\newcommand{\p}{{\mathbf p}}
%\newcommand{\0}{{\mathbf 0}}
%\newcommand{\1}{{\mathbf 1}}
%\newcommand{\N}{{\mathbb N}}
%\newcommand{\Z}{{\mathbb Z}}
%\newcommand{\PP}{{\mathbb P}}
%\newcommand{\Q}{{\mathcal Q}}
%\newcommand{\R}{{\mathbb R}}
%\newcommand{\C}{{\mathbb C}}
%\renewcommand{\H}{{\mathcal H}}
%\newcommand{\V}{{\mathcal V}}
%\newcommand{\T}{{\mathcal T}}
\newcommand{\M}{{\mathcal M}}
%\newcommand{\A}{{\mathcal A}}
%\newcommand{\K}{{\mathcal K}}
%\newcommand{\Lat}{{\mathcal L}}
%\renewcommand{\L}{{\mathcal L}}
%\renewcommand{\S}{{\mathcal S}}
%\newcommand{\IN}{\mathsf{in}}
%%\newcommand{\conv}{\operatorname{conv}}
%\newcommand{\hilb}{\operatorname{Hilb}}
%\newcommand{\wordlength}{\operatorname{w}}
%\newcommand{\growth}{\operatorname{S}}
%\newcommand{\cone}{\operatorname{cone}}
%\newcommand{\aff}{\operatorname{aff}}
%\newcommand{\E}{\operatorname{Ehr}}
%\newcommand{\BP}{\operatorname{BiPyr}}
%\renewcommand{\vert}{\operatorname{vert}}
%\newcommand{\triang}{\operatorname{triang}}
%\newcommand{\total}{\operatorname{total}}
\newcommand{\rank}{\text{rank}}
\newcommand{\diag}{\operatorname{diag}}

\DeclareMathOperator{\EM}{EM}

%\def\th{^{\text{th}}}
%%%%%%%%%%%%%%%%%%%%%%%%%%%%%%%%%%%%%%%

%\setlength{\unitlength}{.05cm}

\newcommand{\ojo}[1]{{\sffamily\bfseries\boldmath[#1]}}
%%%%%%%%%%%%%%%%%%%%%%%%%%%%%%%%%%%%%%%%%
\newcommand{\iid}{\stackrel{\text{iid}}{\sim}}
\title{R Code for Mini Project 2}
\author{Daniel Moss}
\begin{document}
\maketitle
\section{Global notation}
Unless otherwise specified, we adopt the following conventions:
\begin{itemize}
	\item $n$ is the number of nodes in the model
	\item $N$ is the number of iterations
\end{itemize}

\section{General Ising Model code}
\subsection{IsingDesign.R}
Input $n$. Output a Design Matrix which is $2^n$ by $n$
\subsection{IsingDesignQ.R}
Input $n$. Output a Design Matrix which is $2^n$ by $\frac{n(n+1)}{2}$ whose columns correspond to  $X_1,X_2,\dots,X_n,X_1X_2,\dots,X_1X_n,X_2X_3,\dots,X_2X_n,\dots X_{n-1}X_n$.
\subsection{IsingCounts.R}
Outputs a `response vector' of length $2^n$ for use when fitting GLMs (or otherwise) counting each sign pattern, in the usual binary ordering. *Input dataset, though this currently generates one using Gibbs sampling, so input $N$*
\section{Sampling from Ising Models}
\subsection{DirectSampler.R}
Input $N,n$. Gives $N$ direct samples from the Ising model of size $n$ (i.e. by computing the pmf). Currently lambda, mu inputs are manually changed inside the function, and are set to be normal distributions with a given seed.
\subsection{GibbsSampler.R}
As above, but sample is produced via Gibbs sampling.
\section{Equality constraints}
\subsection{FlatteningRank4.R}
Input a four by four matrix to compute the flattening rank.
\subsection{FlatteningRank.R}
Work in progress, will compute flattening rank of $n$ by $n$ matrix.
\section{Fitting MTP$_2$ Ising Models}
\subsection{glmIsing.R}
Input $n$ and $N$. Generates data using IsingCounts.R and regresses this vector against the output of IsingDesignQ.R using a Poisson GLM (log link). Note that on the log scale, using the counts as opposed to the normalized counts only affects the constant.
\subsection{IsingIPS.R}
Implements the algorithm of \cite{lauritzen2019total}. Input sample first and second moments, dimension of model (not really needed) and tolerance. Requires a number of subroutines detailed below.
\subsubsection{CanonicalJ.R}
Inputs $i,j$ and $Y$, the current estimate of the mass function. Outputs $J_{ij}$ (as defined in the reference).
\subsubsection{Delta.R}
Input $i,j,Y,x,M$ and returns $\Delta_{ij}$ (as defined in the reference).
\subsubsection{LambdaStar.R}
Input $i,j,Y,x,M$ and optionally $J_{ij}$ (will probably remove this input). Output $\lambda^*$ (as defined in the reference).
\subsubsection{FirstMoment.R}
Finds the first moment $\mu$ of an inputted mass function (where element $i$ of the pmf vector corresponds to the $i^{th}$ sign pattern when written in the binary ordering).
\subsubsection{SecondMoment.R}
As above but computes second moment $\Xi$.

\bibliographystyle{plain}
\bibliography{refs}
\end{document}